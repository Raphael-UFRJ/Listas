\documentclass[12pt]{article}

\usepackage{sbc-template}
\usepackage{graphicx,url}
\usepackage[utf8]{inputenc}
\usepackage[brazil]{babel}
\usepackage{amsmath}  % Pacote para fórmulas matemáticas
\usepackage{algorithm}
\usepackage{algpseudocode}
\usepackage{float}
\usepackage{caption}
%\usepackage[latin1]{inputenc}  

\sloppy

\title{Análise da Distribuição de Probabilidade do Custo de Desenvolvimento pelo Governo\\ Cálculo de Contagem de Função e Seus Riscos}

\author{Raphael M. S. Jesus\inst{1}}

\address{Programa de Pós-Graduação em Informática\\ Universidade Federal do Rio de Janeiro (UFRJ)\\
  Av. Athos da Silveira Ramos, 274 Prédio do CCMN - Cidade Universitária
Ilha do Fundão \\ Rio de Janeiro, Brasil CEP:21941-916
\nextinstitute
   Instituto Tércio Pacitti de Aplicações e Pesquisas Computacionais\\
  Universidade Federal do Rio de Janeiro (UFRJ)
  \email{raphael.mauricio@gmail.com}
}

\begin{document} 

\maketitle

\begin{abstract}
  Este artigo visa analisar o uso de pontos de função em projetos do governo e estabelecer um cálculo padrão para o valor dos projetos. O foco principal é estabelecer um padrão para o cálculo mais próximo do que seria o ideal para um projeto com os pontos determinados. A análise inclui simulações de Monte Carlo e o uso de distribuições probabilísticas para estimar custos e riscos.
\end{abstract}
     
\begin{resumo} 
  Este artigo visa analisar o uso de pontos de função em projetos do governo e estabelecer um cálculo padrão para o valor dos projetos. O foco principal é estabelecer um padrão para o cálculo mais próximo do que seria o ideal para um projeto com os pontos determinados. A análise inclui simulações de Monte Carlo e o uso de distribuições probabilísticas para estimar custos e riscos.
\end{resumo}

\section{Introdução}

O uso de pontos de função como uma métrica para estimar o tamanho e o esforço de desenvolvimento de software tem sido amplamente adotado tanto na indústria quanto no setor público. No contexto governamental, é crucial que as estimativas de custo sejam precisas e reflitam os riscos associados a variações nos preços e na complexidade técnica dos projetos. Este trabalho visa estabelecer um padrão para a estimativa de custos de desenvolvimento de software em projetos governamentais, utilizando simulações de Monte Carlo para analisar a distribuição de probabilidade dos custos. As fórmulas e métodos utilizados são baseados no \textit{Roteiro de Métricas de Software do SISP} v2.3 \cite{sisp}.

\section{Metodologia}

\subsection{Cálculo de Pontos de Função}

Os Pontos de Função (PF) são uma métrica de tamanho funcional para sistemas de software. O cálculo é realizado através das seguintes etapas, conforme descrito no \textit{Roteiro de Métricas de Software do SISP} v2.3 \cite{sisp}:

1. Identificação e classificação dos componentes do sistema (Entradas Externas, Saídas Externas, Consultas Externas, Arquivos Lógicos Internos e Arquivos de Interface Externa).

2. Atribuição de pesos a cada componente com base na sua complexidade.

3. Aplicação de um fator de ajuste que leva em consideração características gerais do sistema.

As fórmulas para o cálculo dos Pontos de Função são:

\begin{equation}
PF = (ILF \times \text{peso}_\text{ILF}) + (EIF \times \text{peso}_\text{EIF}) + (EI \times \text{peso}_\text{EI}) + (EO \times \text{peso}_\text{EO}) + (EQ \times \text{peso}_\text{EQ})
\end{equation}

\begin{equation}
PF_{\text{ajustado}} = PF \times (0.65 + 0.01 \times \sum_{i=1}^{14} F_i)
\end{equation}

Onde:
\begin{itemize}
    \item \textbf{ILF}: Arquivos Lógicos Internos
    \item \textbf{EIF}: Arquivos de Interface Externa
    \item \textbf{EI}: Entradas Externas
    \item \textbf{EO}: Saídas Externas
    \item \textbf{EQ}: Consultas Externas
    \item \textbf{F$_i$}: Características gerais do sistema (0 a 5)
    \item \textbf{peso$_x$}: Peso de cada componente com base na complexidade
\end{itemize}

\section{Algoritmo de Cálculo de Pontos de Função}

O algoritmo a seguir calcula os Pontos de Função ajustados conforme as fórmulas definidas pelo \textit{Roteiro de Métricas de Software do SISP} v2.3.

\begin{algorithm}[H]
\caption{Calcular Pontos de Função Ajustados}
\begin{algorithmic}[1]
\Procedure{calcularPontosFuncao}{$ILF, EIF, EI, EO, EQ, pesos, ajusteComplexidade$}
    \State $PF \gets (ILF \times pesos[ILF]) + (EIF \times pesos[EIF]) +$
    \State \hspace{3.5em} $(EI \times pesos[EI]) + (EO \times pesos[EO]) + (EQ \times pesos[EQ])$
    \State $PF\_ajustado \gets PF \times (0.65 + 0.01 \times ajusteComplexidade)$
    \State Escrever("Pontos de Função ajustados: ", $PF\_ajustado$)
\EndProcedure
\end{algorithmic}
\end{algorithm}


\subsection{Simulação de Monte Carlo}

A simulação de Monte Carlo é uma técnica utilizada para entender o impacto do risco e da incerteza em modelos e previsões. No contexto deste trabalho, utilizamos a simulação de Monte Carlo para estimar os custos de desenvolvimento de software a partir de pontos de função, considerando variáveis como a experiência da equipe e a complexidade técnica.

\begin{algorithm}[H]
\caption{Calcular custo usando Monte Carlo}
\begin{algorithmic}[1]
\Procedure{calcular\_custo\_mc}{num\_pontos\_funcao, min\_preco, mode\_preco, max\_preco, experiencia\_equipe, complexidade\_tecnica, n\_simulacoes}
    \State custos $\leftarrow$ vetor de tamanho $n\_simulacoes$
    \For{$i = 1$ to $n\_simulacoes$}
        \State preco $\leftarrow$ rtriangle(1, a = min\_preco, b = max\_preco, c = mode\_preco)
        \State custo $\leftarrow$ preco $\times$ num\_pontos\_funcao $\times$ experiencia\_equipe $\times$ complexidade\_tecnica
        \State custos[i] $\leftarrow$ custo
    \EndFor
    \State \Return custos
\EndProcedure
\end{algorithmic}
\end{algorithm}

\subsection{Parâmetros Utilizados}

Para as simulações, foram utilizados os seguintes parâmetros baseados na análise de contratos governamentais:

\begin{itemize}
    \item \textbf{Número de Pontos de Função:} 500
    \item \textbf{Experiência da Equipe:} 1.1 (equivalente a uma equipe com experiência moderada)
    \item \textbf{Complexidade Técnica:} 1.2 (alta complexidade técnica)
    \item \textbf{Número de Simulações:} 10.000
\end{itemize}

Os valores para os preços mínimo, máximo e modal foram derivados de contratos reais, conforme tabela a seguir:

\begin{table}[H]
\centering
\caption{Estimativas de Custos por Tipo de Desenvolvimento}
\resizebox{\textwidth}{!}{%
\begin{tabular}{|l|c|c|c|}
\hline
\textbf{Especificação} & \textbf{Custo Mínimo (R\$)} & \textbf{Custo Máximo (R\$)} & \textbf{Custo Modal (R\$)} \\
\hline
Desenvolvedor JAVA & 186.887,90 & 393.745,98 & 290.000,00 \\
Desenvolvedor PHP & 186.887,90 & 393.745,98 & 290.000,00 \\
Desenvolvedor Python & 186.887,90 & 393.745,98 & 290.000,00 \\
Desenvolvedor Mobile & 186.887,90 & 393.745,98 & 290.000,00 \\
Desenvolvedor Outras Linguagens & 186.887,90 & 393.745,98 & 290.000,00 \\
Desenvolvimento Manutenção de Sistema Legado & 186.887,90 & 393.745,98 & 290.000,00 \\
Desenvolvedor JAVA para Correções e Novos Módulos & 186.887,90 & 393.745,98 & 290.000,00 \\
Desenvolvedor PHP para Correções e Novos Módulos & 186.887,90 & 393.745,98 & 290.000,00 \\
Desenvolvedor Python para Correções e Novos Módulos & 186.887,90 & 393.745,98 & 290.000,00 \\
Desenvolvedor Mobile para Correções e Novos Módulos & 186.887,90 & 393.745,98 & 290.000,00 \\
Desenvolvedor Outras Linguagens para Correções e Novos Módulos & 186.887,90 & 393.745,98 & 290.000,00 \\
\hline
\end{tabular}
}
\label{tab:estimativas_custos}
\end{table}

\section{Resultados e Discussão}

Os resultados da simulação de Monte Carlo fornecem uma distribuição de probabilidade dos custos de desenvolvimento, considerando os parâmetros definidos.

\begin{figure}[H]
\centering
\includegraphics[width=0.8\textwidth]{distribuicao_custos.png}
\caption{Distribuição de Probabilidade dos Custos de Desenvolvimento}
\label{fig:distribuicao_custos}
\end{figure}

\subsection{Análise de Riscos}

Para avaliar o risco de ultrapassar um determinado limite de custo, definimos um limite superior e calculamos a probabilidade de ultrapassagem usando a simulação de Monte Carlo. Esta análise permite que gestores tomem decisões informadas sobre a alocação de recursos e o gerenciamento de riscos nos projetos de software.

\section{Conclusão}

A utilização de Pontos de Função, combinada com simulações de Monte Carlo, oferece uma metodologia robusta para estimar custos e avaliar riscos em projetos de software no setor governamental. A adoção de padrões para cálculo e análise de custos pode contribuir significativamente para a eficiência e transparência na gestão de projetos públicos.

\section{Referências}

\begin{thebibliography}{9}
\bibitem{sisp}
Roteiro de Métricas de Software do SISP • Versão 2.3. Disponível em: \Url{https://www.gov.br/governodigital/pt-br/estrategias-e-governanca-digital/sisp/documentos/arquivos}. Acesso em: [22/06/2024].
\end{thebibliography}

\end{document}
